\documentclass{exam}
\usepackage[utf8]{inputenc}
\usepackage[margin=1in]{geometry}
\usepackage{amsmath}
\usepackage{siunitx}
\usepackage{graphicx}
\usepackage{multicol}
\usepackage{etoolbox}
\usepackage{intcalc}
\usepackage{framed}
\usepackage{tabu}
\usepackage{tabularx}
\newcommand{\match}[2]{
	\begin{tabularx}{\textwidth}{r X}
		\fillin[#1][0.5 in] & #2
	\end{tabularx}}
\newcounter{wbcount}
\newcommand{\wbelem}[1]{\stepcounter{wbcount}
	\textbf{\Alph{wbcount}} & {#1}
	\ifnumequal{0}{\intcalcMod{\value{wbcount}}{\wbcolsize}}
		{\\}
		{&}
}
\newenvironment{wordbank}[1]{
	\renewcommand*{\arraystretch}{1.5} \def\wbcolsize{#1}
	\begin{center} \begin{framed}
	\begin{tabu} to \textwidth {*{#1}{X[1,l] X[5,l]}}
	}
	{\end{tabu} \end{framed} \end{center}
}
\setlength\answerclearance{3 pt}

\usepackage{xcolor}
\newcommand{\choiceblank}[1]{
	\ifprintanswers \underline{\ \ #1\ \ }
	\else \underline{\hspace{0.40 in}}
	\fi
	\vspace{0.05 in}
}
\newcommand{\fixcolspacing}{\vspace{0pt plus 1filll}\mbox{}}
\renewcommand{\solutiontitle}{}
\unframedsolutions
\SolutionEmphasis{\color{violet}}

\printanswers

\begin{document}
\par\noindent \textbf{\large  Matching}
	\raggedcolumns
	\begin{multicols}{5}
	\begin{enumerate}
	\setcounter{enumi}{0}
	\item \choiceblank{J}
	\item \choiceblank{F}
	\item \choiceblank{A}
	\item \choiceblank{E}
	\item \choiceblank{C}
	\item \choiceblank{G}
	\item \choiceblank{D}
	\item \choiceblank{I}
	\item \choiceblank{B}
	\item \choiceblank{H}
	\item \choiceblank{G}
	\item \choiceblank{K}
	\end{enumerate}
	\fixcolspacing
	\end{multicols}
\newpage
\par\noindent \textbf{\large  True/false questions}
\begin{questions}
\setcounter{question}{12}
	\question\match{T}{A new section will create a page break.}
	\question\match{T}{The true/false section is just like the matching, except no word bank is created}
	\question\match{T}{And instead of lettering the choices in the word bank, the answer is either `T' or `F'}
	\question\match{F}{Mars is blue}
	\question\match{F}{America was founded in 1634}
	\question\match{T}{This \LaTeX\ tool is really cool and useful}
\end{questions}
\par\noindent \textbf{\large  The title is Multiple Choice}
	\raggedcolumns
	\begin{multicols}{5}
	\begin{enumerate}
	\setcounter{enumi}{18}
	\item \choiceblank{C}
	\item \choiceblank{A}
	\item \choiceblank{A}
	\item \choiceblank{C}
	\item \choiceblank{B}
	\item \choiceblank{B}
	\item \choiceblank{B}
	\item \choiceblank{A}
	\item \choiceblank{A}
	\end{enumerate}
	\fixcolspacing
	\end{multicols}
\newpage
\par\noindent \textbf{\large  Free Response}
\begin{questions}
\setcounter{question}{27}
\question A question can have parts and subparts
	\begin{parts}
	\part Here is a part
		\begin{solution}[20 pt]
		Here is its solution
		\end{solution}
	\part Here is another part
		\begin{subparts}
		\subpart Its first subpart
			\begin{solution}[20 pt]
			and respective solution
			\end{solution}
		\subpart Next subpart
			\begin{solution}[20 pt]
			and respective solution
			\end{solution}
		\end{subparts}
	\part Last part
		\begin{solution}[20 pt]
		solution
		\end{solution}
	\end{parts}
\question You can put math here: $y=mx+b$ or alternatively \[i\hbar \frac{\partial \Psi}{\partial t} = -\frac{\hbar^2}{2m}\frac{\partial^2 \Psi}{\partial x^2} + V \Psi\] if you like this better.
	\begin{solution}[20 pt]
	solution can have $math$ as well
	\end{solution}
\question You can include commands from packages imported in the header. These come from the physics package: $\laplacian\div\grad\dv{x}$
	\begin{solution}[20 pt]
	and this comes from the hyperref package: \url{google.com}
	\end{solution}
\question Quotation ``marks'' and \% symbols are preserved
	\begin{solution}[20 pt]
	solution
	\end{solution}
\question[3]  Indicate point values by preceding questions with curly bracketed numbers
	\begin{parts}
	\part question1
		\begin{solution}[20 pt]
		 And the number of lines the solution will take up
		\end{solution}
	\part question2
		\begin{solution}[100 pt]
		If solution size is not explicitly given, it`ll try to guess based on the solution you've written. If you write a really long solution, then it will try to allocate more space for the student to match. But sometimes you want to write a long, detailed solution even though the expected solution may not be as long, so it's better to specify the expected number of lines.
		\end{solution}
	\end{parts}
\question You can also directly put some \LaTeX\ commands in here
	\begin{solution}[20 pt]
	and they \textit{should} work
	\end{solution}
		\newpage
\end{questions}
\begin{questions}
\setcounter{question}{33}
\question Back to some sweet
	\begin{parts}
	\part free
		\begin{solution}[20 pt]
		free
		\end{solution}
	\part response
		\begin{solution}[20 pt]
		response
		\end{solution}
	\part questions!
		\begin{solution}[20 pt]
		questions
		\end{solution}
	\end{parts}
\question By default, no answer sheet is generated at the end of the exam. You must use the ans option to specify if you want an answer sheet for a given question module
	\begin{solution}[20 pt]
	ok
	\end{solution}
\question But of course, every question module shows up in the answer key
	\begin{solution}[20 pt]
	of course
	\end{solution}
\end{questions}
\end{document}