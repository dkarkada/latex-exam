\documentclass{exam}
\usepackage[utf8]{inputenc}
\usepackage[margin=1in]{geometry}
\usepackage{amsmath}
\usepackage{siunitx}
\usepackage{graphicx}
\usepackage{multicol}
\usepackage{etoolbox}
\usepackage{intcalc}
\usepackage{framed}
\usepackage{tabu}
\usepackage{tabularx}
\newcommand{\match}[2]{
	\begin{tabularx}{\textwidth}{r X}
		\fillin[#1][0.5 in] & #2
	\end{tabularx}}
\newcounter{wbcount}
\newcommand{\wbelem}[1]{\stepcounter{wbcount}
	\textbf{\Alph{wbcount}} & {#1}
	\ifnumequal{0}{\intcalcMod{\value{wbcount}}{\wbcolsize}}
		{\\}
		{&}
}
\newenvironment{wordbank}[1]{
	\renewcommand*{\arraystretch}{1.5} \def\wbcolsize{#1}
	\begin{center} \begin{framed}
	\begin{tabu} to \textwidth {*{#1}{X[1,l] X[5,l]}}
	}
	{\end{tabu} \end{framed} \end{center}
}
\setlength\answerclearance{3 pt}

\usepackage{xcolor}
\newcommand{\choiceblank}[1]{
	\ifprintanswers \underline{\ \ #1\ \ }
	\else \underline{\hspace{0.40 in}}
	\fi
	\vspace{0.05 in}
}
\newcommand{\fixcolspacing}{\vspace{0pt plus 1filll}\mbox{}}
\renewcommand{\solutiontitle}{}
\unframedsolutions
\SolutionEmphasis{\color{violet}}


\pagestyle{head}
\header{UT Invitational, Fall 2018}{Astronomy C - Page \thepage}{Team Number:\kern .5 in}
\headrule
\printanswers

\begin{document}
\section*{Answer Key}
	\raggedcolumns
	\begin{multicols}{5}
	\begin{enumerate}
	\setcounter{enumi}{0}
	\item \choiceblank{G}
	\item \choiceblank{O}
	\item \choiceblank{D}
	\item \choiceblank{F}
	\item \choiceblank{I}
	\item \choiceblank{N}
	\item \choiceblank{C}
	\item \choiceblank{K}
	\item \choiceblank{A}
	\item \choiceblank{B}
	\item \choiceblank{E}
	\item \choiceblank{L}
	\item \choiceblank{M}
	\item \choiceblank{H}
	\item \choiceblank{J}
	\end{enumerate}
	\end{multicols}
	\raggedcolumns
	\begin{multicols}{5}
	\begin{enumerate}
	\setcounter{enumi}{15}
	\item \choiceblank{A}
	\item \choiceblank{B}
	\item \choiceblank{B}
	\item \choiceblank{C}
	\item \choiceblank{B}
	\item \choiceblank{C}
	\item \choiceblank{A}
	\item \choiceblank{D}
	\item \choiceblank{C}
	\item \choiceblank{A}
	\item \choiceblank{A}
	\item \choiceblank{A}
	\item \choiceblank{B}
	\item \choiceblank{D}
	\item \choiceblank{B}
	\item \choiceblank{C}
	\item \choiceblank{C}
	\item \choiceblank{C}
	\item \choiceblank{C}
	\item \choiceblank{D}
	\item \choiceblank{D}
	\item \choiceblank{C}
	\item \choiceblank{A}
	\item \choiceblank{A}
	\item \choiceblank{B}
	\item \choiceblank{A}
	\item \choiceblank{C}
	\item \choiceblank{B}
	\item \choiceblank{B}
	\item \choiceblank{B}
	\item \choiceblank{A}
	\item \choiceblank{A}
	\item \choiceblank{D}
	\item \choiceblank{C}
	\item \choiceblank{D}
	\end{enumerate}
	\end{multicols}
	\begin{questions}
	\setcounter{question}{50}
	\question
		\begin{parts}
		\part
			 X-rays have very short wavelengths, while infrared radiation has long wavelengths. Shorter wavelength/higher frequency light means more energy.
		\part
			 Binary systems with black holes tend to form accretion disks around the black hole. The accretion disk gets very hot due to internal friction and viscosity. As a result, it radiates X-rays as blackbody radiation.
		\part
			 Light is only ``trapped'' once it is within the event horizon. Light which is emitted by the accretion disk is sufficiently far from the event horizon, and can escape.
		\end{parts}
	\question
		 $\dfrac{\num{2.898e-3}}{2.7}=\num{1.07e-3}\;\si{m}=0.00107$ m
	\question
		\begin{parts}
		\part
			 $d\approx\frac{1}{\theta}=75$ parsecs (small angle approximation)
		\part
			 $0.05/5=0.01$ arcseconds/year
		\part
			 $.01 *75 * \dfrac{1\;\si{\radian}}{206265\;\mathrm{arcsec}} * \dfrac{3.08e13\;\mathrm{km}}{\mathrm{pc}}* \dfrac{1\;\mathrm{yr}}{3.15e7\;\mathrm{s}}=3.6$ km/s (small angle approximation)
		\part
			 $\dfrac{\Delta\lambda}{\lambda_0}*c = \dfrac{0.02}{656.28}*\num{2.99e5}=9.1$ km/s
		\part
			 $\sqrt{3.6^2+9.1^2}=9.8$ km/s
		\part
			 $\dfrac{v}{H_0}=\dfrac{9.1}{70}=0.13\ \mathrm{Mpc}=130,600$ pc
		\part
			 When calculating the distance using Hubble’s Law, the astronomer is assuming that the object is quite far away and the majority of its recessional velocity is coming from the expansion of the universe. However, this star is undoubtedly within our own galaxy (it is very hard to measure the properties of distant stars accurately and parallax is very accurate for relatively short distances). Its motion through the galaxy and gravitational interactions with nearby objects will be much more significant than the recessional velocity caused by the expansion of the universe. As a result, the distance obtained using parallax measurements will be much more accurate than the one found by applying Hubble’s Law.
		\end{parts}
	\question
		\begin{parts}
		\part
			 In these units, most of the constant factors drop out, leaving $P^2=\dfrac{1}{M}a^3$ so $a=(22^2*\num{2e6})^{1/3}=1250$ AU
		\part
			 Using (a variant of) the vis-viva equation, $v=\sqrt{\dfrac{GM}{a}\left(\dfrac{2}{r/a}-1\right)}=\num{3.0e6}$ m/s
		\part
			 $e=\dfrac{r_{max}-r_{min}}{r_{max}+r_{min}}=0.5$
		\end{parts}
	\question
		\begin{parts}
		\part
			 M82 (partial credit for saying galaxy).
		\part
			 Dense clouds of cold gas and dust is blocking the light.
		\part
			 Very hot gas being ejected from the galaxy is glowing bright in far UV and X-ray.
		\end{parts}
	\end{questions}
	\begin{questions}
	\setcounter{question}{55}
	\question
		 Compare theoretical models of X-ray emission (for a neutron star) with the observed X-ray emission, and show that they don't match.
	\question
		 Since the mass is above the TOV limit, if the neutron star weren't rotating then it would immediately collapse to a black hole. However, a massive neutron star which is rotating sufficiently fast can resist collapse via the centrifugal ``force''.
	\question
		 As a neutron star spins, its strong magnetic field interacts with the surrounding plasma, accelerating the charged particles, which then emit light via synchrotron emission. This energy is taken from the rotational energy of the neutron star; consequently, the emission of synchrotron light actually slows down the neutron star.
	\question
		 The X-ray emission from spin-down emission would deposit energy into the local interstellar medium, which would result in a brightening. (As an aside: the delay is caused by the time it takes for the energy to propagate through the shock to the shock front.)
	\question
		 Full credit for mentioning either relativistic shocks or relativistic jets.
	\end{questions}
	\begin{questions}
	\setcounter{question}{60}
	\question
		 “$U-B$” is the difference between the galaxies’ $U$ and $B$ magnitudes and “$B-V$” is the difference between the galaxies’ $B$ and $V$ magnitudes.
	\question
		 For any color index (including the two in this problem), a lower number signifies being “bluer”. The $y$-axis decreases going up, while the $x$-axis decreases going to the left. As a result, the “bluest” galaxy would be something in the top left corner, where letter A is.
	\question
		 Type 1 Seyfert galaxies have two sets of emission lines, superposed on one another. One set of lines is characteristic of a low-velocity gas and are referred to as the “narrow lines”. A second set of “broad lines” are seen, corresponding to higher velocities. Type 2 Seyfert galaxies differ from Seyfert 1 galaxies in that the broad lines are weaker/absent in type 2 spectra.
	\end{questions}
	\begin{questions}
	\setcounter{question}{63}
	\question
		 The non-interacting galaxies have a comparable scatter to the “normal” galaxies of the Hubble Atlas, but the interacting galaxies have a lot more variation. As a result, the interacting galaxies must be the cause of the large amounts of variation in the “peculiar” galaxies’ colors.
	\question
		 Galaxy interactions can lead to high velocity collisions and shock fronts. Both should be effective in compressing gas to high densities and ultimately leading to star formation.
	\end{questions}
\end{document}