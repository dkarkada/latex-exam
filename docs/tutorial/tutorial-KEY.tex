\documentclass{exam}
\usepackage[utf8]{inputenc}
\usepackage[margin=1in]{geometry}
\usepackage{amsmath}
\usepackage{siunitx}
\usepackage{graphicx}
\usepackage{multicol}
\usepackage{etoolbox}
\usepackage{intcalc}
\usepackage{framed}
\usepackage{tabu}
\usepackage{tabularx}
\newcommand{\match}[2]{
	\begin{tabularx}{\textwidth}{r X}
		\fillin[#1][0.5 in] & #2
	\end{tabularx}}
\newcounter{wbcount}
\newcommand{\wbelem}[1]{\stepcounter{wbcount}
	\textbf{\Alph{wbcount}} & {#1}
	\ifnumequal{0}{\intcalcMod{\value{wbcount}}{\wbcolsize}}
		{\\}
		{&}
}
\newenvironment{wordbank}[1]{
	\renewcommand*{\arraystretch}{1.5} \def\wbcolsize{#1}
	\begin{center} \begin{framed}
	\begin{tabu} to \textwidth {*{#1}{X[1,l] X[5,l]}}
	}
	{\end{tabu} \end{framed} \end{center}
}
\setlength\answerclearance{3 pt}

\usepackage{xcolor}
\newcommand{\choiceblank}[1]{
	\ifprintanswers \underline{\ \ #1\ \ }
	\else \underline{\hspace{0.40 in}}
	\fi
	\vspace{0.05 in}
}
\newcommand{\fixcolspacing}{\vspace{0pt plus 1filll}\mbox{}}
\renewcommand{\solutiontitle}{}
\unframedsolutions
\SolutionEmphasis{\color{violet}}

\usepackage{physics}
\usepackage{hyperref}

\pagestyle{head}
\header{Class}{Exam title - Page \thepage}{Student ID:\kern .5 in}
\headrule
\printanswers

\begin{document}
\section*{Answer Key}
	\raggedcolumns
	\begin{multicols}{5}
	\begin{enumerate}
	\setcounter{enumi}{0}
	\item \choiceblank{F}
	\item \choiceblank{A}
	\item \choiceblank{H}
	\item \choiceblank{L}
	\item \choiceblank{C}
	\item \choiceblank{E}
	\item \choiceblank{G}
	\item \choiceblank{D}
	\item \choiceblank{G}
	\item \choiceblank{J}
	\item \choiceblank{K}
	\item \choiceblank{B}
	\item \choiceblank{I}
	\end{enumerate}
	\fixcolspacing
	\end{multicols}
	\raggedcolumns
	\begin{multicols}{5}
	\begin{enumerate}
	\setcounter{enumi}{13}
	\item \choiceblank{T}
	\item \choiceblank{T}
	\item \choiceblank{T}
	\item \choiceblank{F}
	\item \choiceblank{F}
	\item \choiceblank{T}
	\end{enumerate}
	\fixcolspacing
	\end{multicols}
	\raggedcolumns
	\begin{multicols}{5}
	\begin{enumerate}
	\setcounter{enumi}{19}
	\item \choiceblank{C}
	\item \choiceblank{D}
	\item \choiceblank{A}
	\item \choiceblank{A}
	\item \choiceblank{B}
	\item \choiceblank{B}
	\item \choiceblank{A}
	\item \choiceblank{A}
	\item \choiceblank{A}
	\end{enumerate}
	\fixcolspacing
	\end{multicols}
	\begin{questions}
	\setcounter{question}{28}
	\question
		\begin{parts}
		\part
			Here is its solution
		\part
			\begin{subparts}
			\subpart
				and respective solution
			\subpart
				and respective solution
			\end{subparts}
		\part
			solution
		\end{parts}
	\question
		\begin{parts}
		\part
			nice
		\part
			ok
		\end{parts}
	\question
		solution can have $math$ as well
	\question
		and this comes from the hyperref package: \url{google.com}
	\question
		solution
	\question
		\begin{parts}
		\part
			 And the number of lines the solution will take up
		\part
			If solution size is not explicitly given, it'll try to guess based on the solution you've written. If you write a really long solution, then it will try to allocate more space for the student to match. But sometimes you want to write a long, detailed solution even though the expected solution may not be as long, so it's better to specify the expected number of lines.
		\end{parts}
	\question
		and they \textit{should} work
	\end{questions}
	\begin{questions}
	\setcounter{question}{35}
	\question
		\begin{parts}
		\part
			 free
		\part
			 response
		\part
			 questions
		\end{parts}
	\question
		 ok
	\question
		 of course
	\end{questions}
\end{document}